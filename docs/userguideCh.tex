\chapter*{User guide}
\addcontentsline{toc}{chapter}{User guide}

\section*{Usage}
\addcontentsline{toc}{section}{Usage}

For usage you need:

\begin{itemize}
	\item Logisim
	\item External terminal with ability to connect to UART via telnet protocol.
	\item Product itself (all circuits and main programm image)
\end{itemize}

\begin{enumerate}
	\item Open \texttt{main.circ}.
	\item Load \texttt{main.img} into RAM chip.
	\item Turn on ticks.
	\item Connect to 7241 port of localhost via telnet:
		\begin{itemize}
			\item \textbf{Linux} just use \texttt{telnet localhost 7241} command.
			\item \textbf{Windows} use PuTTY, but turn negotiations mode to "passive" (in telnet settings).
		\end{itemize}
	\item You will see hello message, type "h" to see list of commands. Enjoy!
\end{enumerate}

\section*{Building}
\addcontentsline{toc}{section}{Building}

For building purposes you need:

\begin{itemize}
	\item \href{https://github.com/Proletcultist/cdm-devkit-macro-improvements}{This} fork of CdM-devkit
	\item Some make implementation (such as GNU make)
\end{itemize}

You can build all with \texttt{make all}. Also, you can build only tests with \texttt{make tests} or only main programm with \texttt{make main}. To add new test just add any .asm file you want to src/Tests directory (your test should provide some implementation for main function).

For debugging purposes use \texttt{make $<$test name$>.debug$}. You should provide name of test from src/Tests directory (without extension .asm). As a result, you will have .img in bin/Tests and debug info file in debug/
