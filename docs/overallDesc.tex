\chapter*{Overall description}
\addcontentsline{toc}{chapter}{Overall description}

\section*{Product features}
\addcontentsline{toc}{section}{Product features}

All developed hardware presented as logisim circuits.

Product realise cellular automaton with 32x32 game field and set of rules, including:

\begin{itemize}
	\item \textbf{Born} - set of integers from 0 to 8, specifying, with what amount of alive neighbours cell will become alive.
	\item \textbf{Surv} - set of integers from 0 to 8, specifying, with what amount of alive neighbours cell will survive.
	\item \textbf{Speed} - integer from 0 to 3, specifying game speed.
\end{itemize}

CdM-16 processor has the ability to communicate with cellular automaton, within this communication processor can read and modify automaton state.

Product allows user to connect to it with external terminal through UART. With this connection user can do following:

\begin{itemize}
	\item Modify game field
	\item Modify rules
	\item Control game flow, start or pause game, trigger single iteration of game
	\item Save and paste templates (some part specified part of game field, stored in one of 5 slots)
\end{itemize}

% TODO: Add DFD diagramm

\section*{Operating environment}
\addcontentsline{toc}{section}{Operating environment}

For running product you need:

\begin{itemize}
	\item Logisim
	\item \href{https://github.com/cdm-processors/cdm-devkit}{CdM-devkit} ver. 0.2.2 or above JAR libraries (cdm16 emulator, banked memory)
	\item UART JAR library
	\item External terminal with ability to connect to UART via telnet protocol.
\end{itemize}

For building purposes you need:

\begin{itemize}
	\item \href{https://github.com/Proletcultist/cdm-devkit-macro-improvements}{This} fork of CdM-devkit
	\item Some of make implementation (such as GNU make)
\end{itemize}
