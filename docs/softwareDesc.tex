\chapter*{Software}
\addcontentsline{toc}{chapter}{Software}

\section*{Memory layout}
\addcontentsline{toc}{section}{Memory layout}

\begin{itemize}
	\item 0x0000 - 0x0080 - IVT
	\item 0xfc56 - SP initial value
	\item 0xfc56 - 0xfee0 - templates array 
	\item 0xfee1 - 0xfeff - input buffer 
	\item 0xff00-0xffff - I/O devices address space \begin{itemize}
		\item 0xff74 - address for trigerrinh clearing of field
		\item 0xff76 - address for triggering one game tick
		\item 0xff78 - address for controlling of interrupts throwing from UART controller
		\item 0xff7a - UART controller (for reading and writing data via UART) 
		\item 0xff7c - matrix controller state register №2 \begin{itemize}
				\item Bits 0-8 - bit string, representing born rule
				\item Bit 9 - play
				\item Bit 10 - reset (write 1 to this bit, to reset field)
				\item Bits 11-12 - speed
		\end{itemize}
		\item 0xff7e - matrix controller state register №1 \begin{itemize}
				\item Bits 0-8 - bit string, representing survive rule
		\end{itemize}
		\item 0xff80-0xffff - game field (for reading and writing of current field state 16 bits at a time) 
	\end{itemize}
\end{itemize}

\section*{Start section}
\addcontentsline{toc}{section}{Start section}

\begin{lstlisting}
main: ext
rsect start
start>
        ldi r0, 0xfc56
        stsp r0
        jsr main
        halt
end
\end{lstlisting}

Start section with start function in it stays for preparing runtime (setting SP initial value) and calling main function.

\section*{Main section}
\addcontentsline{toc}{section}{Main section}

\begin{lstlisting}
rsect main

main>
	ei
	w:
	wait
	br w
	rts
end
\end{lstlisting}

Main function simply enables interrupts and stay in infinite loop with wait in order to wait for interrupt, process it and then return into wait status.

\section*{Matrix driver}
\addcontentsline{toc}{section}{Matrix driver}

Matrix driver is a library, which implements functions, allowing abstraction from low-level interactions with matrix controller. This library has following functions:

\begin{enumerate}
	\item setTo1
		\begin{itemize}
			\item Arguments: \textbf{r0} - $x$, \textbf{r1} - $y$. Coordinates should be in range 0-31.
			\item Sets cell with coordinates $(x, y)$ to alive status
		\end{itemize}
	\item setTo0
		\begin{itemize}
			\item Arguments: \textbf{r0} - $x$, \textbf{r1} - $y$. Coordinates should be in range 0-31.
			\item Sets cell with coordinates $(x, y)$ to dead status
		\end{itemize}
	\item invert
		\begin{itemize}
			\item Arguments: \textbf{r0} - $x$, \textbf{r1} - $y$. Coordinates should be in range 0-31.
			\item Inverts cell status with coordinates $(x, y)$
		\end{itemize}
	\item setRectTo1
		\begin{itemize}
			\item Arguments: \textbf{r0} - $x_{1}$, \textbf{r1} - $y_{1}$, \textbf{r2} - $x_{2}$, \textbf{r3} - $y_{2}$. Coordinates should be in range 0-31 and $x_{1} \leq x_{2}$, $y_{1} \leq y_{2}$.
			\item Sets all cells in rectangle with upper left corner ($x_{1}$, $y_{1}$) and lower right corner ($x_{2}$, $y_{2}$) to alive status
		\end{itemize}
	\item setRectTo0
		\begin{itemize}
			\item Arguments: \textbf{r0} - $x_{1}$, \textbf{r1} - $y_{1}$, \textbf{r2} - $x_{2}$, \textbf{r3} - $y_{2}$. Coordinates should be in range 0-31 and $x1 \leq x2$, $y1 \leq y2$.
			\item Sets all cells in rectangle with upper left corner ($x_{1}$, $y_{1}$) and lower right corner ($x_{2}$, $y_{2}$) to dead status
		\end{itemize}
	\item startGame
		\begin{itemize}
			\item Starts game
		\end{itemize}
	\item pauseGame
		\begin{itemize}
			\item Pauses game
		\end{itemize}
	\item setSurv
		\begin{itemize}
			\item Arguments: \textbf{r0} - bit string, representing survival rule. If bit number n (0-9) is 1, then alive cell with n alive neighbours will survive, otherwise it will die. All bits from 10th to 15th should be 0.
			\item Sets rule "survive"
		\end{itemize}
	\item setBorn
		\begin{itemize}
			\item Arguments: \textbf{r0} - bit string, representing born rule. If bit number n (0-9) is 1, then dead cell with n alive neighbours will becom alive, otherwise it remain dead. All bits from 10th to 15th should be 0.
			\item Sets rule "born"
		\end{itemize}
	\item clear
		\begin{itemize}
			\item Cleares game field
		\end{itemize}
	\item speedDown
		\begin{itemize}
			\item Increments speed, which causes slowdown (because 0 - the fastest speed). If speed is 3, it will change to 0
		\end{itemize}
	\item speedUp
		\begin{itemize}
			\item Decrements speed, which causes speed increae (because 0 - the fastest speed). If speed is 0, it will change to 3
		\end{itemize}
	\item setSpeed
		\begin{itemize}
			\item Arguments: \textbf{r0} - speed value. Bits 0-1 - speed value, other bits should be 0.
			\item Sets speed to specified value
		\end{itemize}
	\item stepOnce
		\begin{itemize}
			\item Do one iteration of game
		\end{itemize}
	\item insertTemplate
		\begin{itemize}
			\item Arguments: \textbf{r0} - slot number, \textbf{r1} - $x$, \textbf{r2} - $y$. Coordinates should be in range 0-31.
			\item Inserts template from specified slot, corrdinates $x$, $y$ specify upper left corner of inserted template.
			\item If inserting was unsuccessful, returns $-1$ in r2
		\end{itemize}
	\item saveTemplate
		\begin{itemize}
			\item Arguments: \textbf{r0} - slot number, \textbf{r1} - $x_{1}$, \textbf{r2} - $y_{1}$, \textbf{r3} - $x_{2}$, \textbf{r4} - $y_{2}$.
			\item Saves area with upper left corner $(x_{1}, y_{1})$ and lower right corner $(x_{2}, y_{2})$ as template to specified slot.
		\end{itemize}
\end{enumerate}

\subsection*{Matrix templates}
\addcontentsline{toc}{subsection}{Matrix templates}

Template is a saved part of game field. It represented as size of template and stream if bits, which represent states of saved cells. There is 5 templates in templates array.

Template structure:

\begin{itemize}
	\item 1 word - header
		\begin{itemize}
			\item bits 0 - 4 - width
			\item bits 5 - 9 - height
		\end{itemize}
	\item Following 64 words - template
\end{itemize}

\section*{Parser library}
\addcontentsline{toc}{section}{Parser library}

Parser library implements function for parsing command line, stored in input buffer in memory, and util functions, helping do this. This function gets command, its arguments, checks all of this for errors, if there is any error, function print meaningful error message, otherwise it calls appropriate function from matrix driver with parsed arguments.

Utilities:

\begin{enumerate}
	\item \textbf{skipSpaces}
		\begin{itemize}
			\item Argument: r0 - pointer to buffer's symbol
			\item Skips all spaces from current symbol
		\end{itemize}
	\item \textbf{strncmp}
		\begin{itemize}
			\item Arguments: r0 - pointer to buffer's symbol, r1 - pointer to command's symbol, r4 - number of symbols
			\item Compares the first n symbols from the current symbol with a command of specified size
			\item Returns $1$ in r4 if strings equal, $0$ - otherwise
		\end{itemize}
	\item \textbf{readUInt}
		\begin{itemize}
			\item Arguments: r0 - pointer to buffer's symbol
			\item Gets a number from the buffer from the current symbol
			\item If reading was successful, returns readen number in r1
			\item If reading was unsuccessful, returns $-1$ in r1
		\end{itemize}
	\item \textbf{readRules}
		\begin{itemize}
			\item Arguments: r0 - pointer to buffer's symbol
			\item Gets rules from the buffer from the current symbol
			\item If reading was successful, returns \textbf{surv} rule in r2 and \textbf{born} rule in r5.
			\item If reading was unsuccessful, returns $-1$ in r1
		\end{itemize}
\end{enumerate}

General algorithm for parsing:

\begin{enumerate}
	\item Call the skipSpaces function
	\item Compare with the command using strncmp
	\item Call skipSpaces again
	\item Then call skipSpaces and readUInt as needed
\end{enumerate}

\section*{UART driver}
\addcontentsline{toc}{section}{UART driver}

Functions:

\begin{enumerate}
	\item \textbf{readFromUART}
		\begin{itemize}
			\item Reads line from UART buffer to input buffer in memory. Maximum line size - 30 characters.
		\end{itemize}
	\item \textbf{writeToUART}
		\begin{itemize}
			\item Arguments: r0 - pointer to string
			\item Writes string to UART
		\end{itemize}
	\item \textbf{freeUART}
		\begin{itemize}
			\item Arguments: r0 - pointer to UART buffer
			\item Reads line from UART, doesn't save it anywhere or processes it anyhow (this is useful, when line exceeds limit, and we need to delete "tail" of this line from UART).
		\end{itemize}
\end{enumerate}

\section*{Interrupt handlers}
\addcontentsline{toc}{section}{Interrupt handlers}

\begin{enumerate}
	\item \textbf{helloPrint}
		\begin{itemize}
			\item Prints welcome message
			\item It's called when UART is connected
		\end{itemize}
	\item \textbf{getLine}
		\begin{itemize}
			\item Reads string from UART buffer and parses it
			\item It's called when UART buffer has symbols
		\end{itemize}
\end{enumerate}

\section*{Commands list}
\addcontentsline{toc}{section}{Commands list}

\begin{enumerate}
	\item \textsf{h} - print help message
	\item \textsf{s1 $x$ $y$} - set cell $(x, y)$ to alive status
	\item \textsf{s0 $x$ $y$} - set cell $(x, y)$ to dead status
	\item \textsf{i $x$ $y$} - invert status of $(x, y)$ cell
	\item \textsf{play} - run game
	\item \textsf{pause} - stop game
	\item \textsf{step} - make one iteration of game
	\item \textsf{rules S$<$digits sequence$>$B$<$digits sequence$>$} - set rules
	\item \textsf{su} - speed up
	\item \textsf{sd} - speed down
	\item \textsf{ss $n$} - set speed to value $n$
	\item \textsf{S1 $x_{1}$ $y_{1}$ $x_{2}$ $y_{2}$} -  Sets all cells in rectangle with upper left corner ($x_{1}$, $y_{1}$) and lower right corner ($x_{2}$, $y_{2}$) to alive status
	\item \textsf{S0 $x_{1}$ $y_{1}$ $x_{2}$ $y_{2}$} -  Sets all cells in rectangle with upper left corner ($x_{1}$, $y_{1}$) and lower right corner ($x_{2}$, $y_{2}$) to dead status
	\item \textsf{save $n$ $x_{1}$ $y_{1}$ $x_{2}$ $y_{2}$} - Save all cells in rectangle with upper left corner ($x_{1}$, $y_{1}$) and lower right corner ($x_{2}$, $y_{2}$) to slot number $n$.
	\item \textsf{load $n$ $x$ $y$} - Copy all cells from slot $n$ to rectangle with upper left corner ($x_{1}$, $y_{1}$)
	\item \textsf{clr} - clear game field.
\end{enumerate}
